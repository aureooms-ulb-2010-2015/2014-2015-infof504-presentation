\section{Clarkson's Algorithm}
\begin{frame}\frametitle{\insertsection}\justifying
\begin{ebox}{Assumptions}
These assumptions can be guaranteed using \(d\cdot\LP{}(d,n)\) time
\cite{fukuda:2015}.
\begin{itemize}
\item[{\color{SpringGreen4}1.}] The polyhedron defined by \(H\) is full-dimensional.
\item[{\color{SpringGreen4}2.}] No inequality is a positive multiple of another one.
\item[{\color{SpringGreen4}3.}] We are given a vertex \(z\) in the interior of the polytope defined by \(H\).
\end{itemize}
\end{ebox}
\pause
\begin{algo}[(Clarkson's algorithm)]
\item[input] \((H, z)\)
\item[1.] Let \(M\) be the empty set.
\item[2.] While \(H \neq \emptyset\)
\item[2.1.] Let \(h \in H\), decide whether \(h\) is redundant w.r.t. \(M\).
\item[2.2.] If so, \(H \gets H \setminus \enum{h}\).
\item[2.3.] Otherwise, use \(h\) to find a \(h^* \in H\) such that \(h^*\) is
nonredundant w.r.t. \(H\). \(H \gets H \setminus \enum{h^*}, M
\gets M \cup \enum{h^*}\).
\item[3.] Output \(M\).
\end{algo}
\end{frame}

\begin{frame}\frametitle{\insertsection}\justifying
\begin{ex}[(Finding \(h^*\))]
\begin{center}
\psset{unit=.05cm}
\begin{pspicture}(0,100)(100,0)
% polytope
\pspolygon*[linecolor=MediumOrchid2!50!white](75,80)(60,20)(38.5,25.5)(33,47)


% half-spaces (hyperplane + direction)

\psset{linecolor=black!20!white}
\psline(0,90)(100,90)
\psline{<-}(85,85)(85,90)
\psline(45,0)(20,100)
\psline{->}(36,37)(41,38.25)
\only<3->{\psset{linecolor=violet}}
\psline(0,20)(100,100)
\psline{->}(50,60)(54,55)
\psset{linecolor=black}

\psset{linecolor=blue}
\uput{2}[0](100,80){{\color{blue}\(h\)}}
\psline(0,80)(100,80)
\psline{<-}(85,75)(85,80)
\psset{linecolor=black}

\only<3->{
\uput{2}[180](0,20){{\color{violet}\(h^*\)}}
}
\psset{linecolor=black}
\psline(0,35)(100,10)
\psline{->}(49,23)(50.3,28)
\psline(55,0)(80,100)
\psline{<-}(58,33.75)(63,32.5)

\only<2->{\psline[linestyle=dashed,dash=3pt 2pt](52,40)(52,80)}

\psdots(52,40)
\uput{2}[0](52,40){\(z\)}
\only<2->{
\psdots(52,80)
\uput{2}[0](52,80){\(x^*\)}
}
\end{pspicture}
\end{center}

\end{ex}
\end{frame}
